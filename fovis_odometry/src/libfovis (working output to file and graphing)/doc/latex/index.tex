\hypertarget{index_introduction}{}\section{Introduction}\label{index_introduction}
Fovis is a visual odometry library that estimates the 3D motion of a camera using a source of depth information for each pixel. It's original implementation is described in the following paper:

\begin{DoxyItemize}
\item Visual Odometry and Mapping for Autonomous Flight Using an RGB-\/D Camera. {\itshape Albert S. Huang, Abraham Bachrach, Peter Henry, Michael Krainin, Daniel Maturana, Dieter Fox, and Nicholas Roy\/} Int. Symposium on Robotics Research (ISRR), Flagstaff, Arizona, USA, Aug. 2011 \href{http://people.csail.mit.edu/albert/pubs/2011-huang-isrr.pdf}{\tt \mbox{[}PDF\mbox{]}}.\end{DoxyItemize}
\hypertarget{index_build_requirements}{}\section{Build requirements}\label{index_build_requirements}
Fovis is intended to be relatively portable. There are two major requirements for building and using the software: \begin{DoxyItemize}
\item \href{http://eigen.tuxfamily.org}{\tt Eigen 3} \item A CPU supporting Intel SSE2.\end{DoxyItemize}
\hypertarget{index_usage_requirements}{}\section{Usage requirements}\label{index_usage_requirements}
For portability reasons, the actual library itself is sensor agnostic and provides no data acquisition capabilities. To use fovis, your program must acquire data on its own and pass it through to the fovis API. Some examples are provided with the source code.

Effective use of fovis for visual odometry requires the following: \begin{DoxyItemize}
\item A source of 8-\/bit grayscale camera images. \item A {\itshape camera calibration\/} for the images that provides an accurate mapping between image pixel coordinates $(u, v)$ and 3D rays $(X, Y, Z)$ in the camera's Cartesian coordinate frame. \item A {\itshape depth source\/} for each image. A depth source must be able to provide a metric depth estimate for as many pixels in the camera image as possible.\end{DoxyItemize}
Fovis provides built-\/in support for the following types of depth sources: \begin{DoxyItemize}
\item An RGB-\/D camera such as the Microsoft Kinect. \item Calibrated stereo cameras.\end{DoxyItemize}
You can also create your own depth sources using the Fovis API and adapt it to other sensor types.\hypertarget{index_getting_started}{}\section{Getting started}\label{index_getting_started}
The best way to get started is to look through the examples provided with the source code in the {\ttfamily examples/} directory.

Next, look through the Fovis C++ API. The primary class of interest is \hyperlink{classfovis_1_1VisualOdometry}{fovis::VisualOdometry}.\hypertarget{index_license}{}\section{License}\label{index_license}
fovis is free software: you can redistribute it and/or modify it under the terms of the GNU General Public License as published by the Free Software Foundation, either version 3 of the License, or (at your option) any later version.

fovis is distributed in the hope that it will be useful, but WITHOUT ANY WARRANTY; without even the implied warranty of MERCHANTABILITY or FITNESS FOR A PARTICULAR PURPOSE. See the GNU General Public License for more details.

A copy of the GNU General Public License is provided with the fovis source code. 